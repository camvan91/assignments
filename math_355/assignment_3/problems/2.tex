\section{Using $\e$-$\delta$ arguments, directly prove the following limits:}
    \subsection{
        \[
            \bm{\lim_{x\to 4} \frac{2x-3}{\sqrt{x-3}} = 5}
        \]
    }
        Let $\e > 0$. 
        Choose $\delta = \min\set{\frac{1}{2}, \e \left(\frac{18 + 21 + \frac{148}{0.5}}{4\sqrt{0.5} + 2.5}\right)^{-1}}$.
        Suppose $0<\abs{x - 4} < \delta$.
        So, 
        $\abs{x - 4} < \frac{1}{2} $,
        $3.5 < x < 4.5$.
        Now, 
        \begin{align*}
            \abs{f(x) - 5}
            &= \abs{\frac{2x-3}{\sqrt{x-3}} - 5}
            = \abs{\frac{2x-3 - 5\sqrt{x-3}}{\sqrt{x-3}}} \\
            &= \abs{\frac{(2x-3)^2 - (5\sqrt{x-3})^2}{(2x-3 + 5\sqrt{x-3})\sqrt{x-3}}} 
            = \abs{\frac{4x^2 - 12x + 9 - 25(x-3)}{(2x-3 + 5\sqrt{x-3})\sqrt{x-3}}} 
            = \abs{\frac{4x^2 - 37x -64}{(2x-3 + 5\sqrt{x-3})\sqrt{x-3}}} \\
            &= \abs{\frac{(4x - 21 - \frac{148}{x-4})(x-4)}{(2x-3 + 5\sqrt{x-3})\sqrt{x-3}}} 
            = \frac{\abs{4x - 21 - \frac{148}{x-4}}\abs{x-4}}{\abs{(2x-3 + 5\sqrt{x-3})\sqrt{x-3}}} \\
            % &= \abs{x-4} \frac{\abs{4x - 21 - \frac{148}{x-4}}}{\abs{(2x-3 + 5\sqrt{x-3})\sqrt{x-3}}} \\
            &< \abs{x-4} \frac{\abs{4(4.5) - 21 - 
                \frac{148}{x-4}}}{\abs{(2(3.5)-3 + 5\sqrt{3.5-3})\sqrt{3.5-3}}} 
            = \abs{x-4} \frac{\abs{4(4.5) - 21 - \frac{148}{x-4}}}{4\sqrt{0.5} + 2.5} \\
            &< \abs{x-4} \frac{18 + 21 + \abs{\frac{148}{x-4}}}{4\sqrt{0.5} + 2.5} 
            < \abs{x-4} \frac{18 + 21 + \frac{148}{0.5}}{4\sqrt{0.5} + 2.5} \\
            &< \e \left(\frac{18 + 21 + \frac{148}{0.5}}{4\sqrt{0.5} + 2.5}\right)^{-1}
                \frac{18 + 21 + \frac{148}{0.5}}{4\sqrt{0.5} + 2.5}
            = \e
        \end{align*}

    \subsection{
        \[
            \bm{\lim_{x\to -1^-}\frac{1}{x^2-1} = \infty}
        \]
    }

        Let $M > 0$.
        Choose $\delta = \min\set{1, \frac{1}{3M}}$.
        Suppose $0 < -1-x < \delta$.
        So, 
        $0< -1-x < 1$,
        $1< -x < 2$,
        $-1 > x > -2$.
        Now,
        \begin{align*}
            f(x) 
            &= \frac{1}{x^2 - 1}\\
            &= \frac{1}{(x+1)(x-1)}\\
            &> \frac{1}{(x+1)(-2-1)}
            = \frac{1}{-(-x-1)(-3)}\\
            &> \frac{1}{-\frac{1}{3M}(-3)}
            = M
        \end{align*}
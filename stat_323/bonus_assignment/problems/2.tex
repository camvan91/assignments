\section{Is academic salary related to the number of years since the highest degree earned? A
    random sample of 52 tenure-track professors was taken. Each professor’s academic
    yearly salary (in dollars) was recorded, as well as the time (in years) since their highest
    degree had been earned (this data is called “Salary” and “Years” respectively in the
    “Bonus Assignment Data” R file).}
    \begin{multicols}{2}
        \subsection{What is the value of the coefficient of determination? Interpret its meaning in this
            context.}
            $r^2 \approx 0.4554$

        \subsection{Write down the least-squares estimate of the model predicting yearly salary from years
            since highest degree earned.}
            $s \approx 390.6y + 17500$

        \subsection{Interpret the slope of this regression equation.}
            For every 1 year increase the amount of time since highest degree earned, 
            there will be a 390.6 dollar increase in salary.

        \subsection{Interpret the y-intercept of this regression equation. Does the y-intercept have a
            “practical” (meaningful) interpretation?}
            It means the starting salary from highest degree earned is 17500 dollars.\\
            Yes

        \subsection{Predict the yearly salary for a tenure-track professor for whom there has been 29
            years since his/her highest degree was earned.}
            $s \approx 390.6451y + 17502.2574$\\
            $= 390.6451(29) + 17502.2574$
            $\approx 28830$

        \subsection{A tenure-track professor for whom there has been 29 years since his/her highest
            degree was earned is actually earning $\$22,450$ yearly. Compute this professor’s
            residual.}
            $\epsilon = s - \hat s \approx 22450 - (390.6451(29) + 17502.2574)$
            $\approx -6381$
            
        \subsection{Examine the normality plot. What condition/assumption does the normality plot check for?
            Does it appear to be met in this case?}
            See figure \ref{fig:2g} on page \pageref{fig:2g}.\\
            The normality plot checks if the data approximates a normal distro.
            In this case, it does, as the dots are close enough to the line.
            And the histogram is close enough to a bell curve.
    \end{multicols}
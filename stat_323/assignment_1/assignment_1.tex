\documentclass[10pt, letterpaper, titlepage]{article}

\usepackage{amsmath}

%Header
\usepackage[margin=1in]{geometry}
\usepackage{fancyhdr}
\setlength{\headheight}{22.54448pt}
\pagestyle{fancy}
\lhead{}
\rhead{Yifeng Pan
     \\UCID: 30063828}

%Change lable to letter from number
\renewcommand{\thesubsection}{\alph{subsection}}

%Evaluate for calc
\newcommand*\eval[3]{\left.#1\right\rvert_{#2}^{#3}}

%Absolute Value
\newcommand\abs[1]{\left|#1\right|}

%Title page
\title{STAT 323 Assignment 1}
\author{Instructor: Claudia Marie Mahler
    \\Name: Yifeng Pan
    \\UCID: 30063828}
\date{Summer 2019}

%mean of x
\newcommand\mx{\overline{x}}

\begin{document}
    \maketitle
    \section[]{
        Let $X$ be a random variable with a density function given by
        \[
            f(x) = 
            \begin{cases} 
                \frac{3}{2}x^2 & \text{for } -1 \leq x \leq 1 \\
                0              & \text{elsewhere}
            \end{cases}
        \]
    }
        \subsection{Find the density function of $Y = 3 - X$.}
            By method of distributon functions:\\
            $X = 3 - Y$. So: 
            \begin{align*}
                -1 \leq &X \leq 1 \\
                -1 \leq &3-Y \leq 1 \\
                -4 \leq &-Y \leq -2 \\
                 2 \leq &Y \leq 4 
            \end{align*}
            $F(y) = P(Y \leq y) = P(3-X \leq y) = P(-X \leq y - 3) = P(X \geq 3 - y)$
            \[
                = \int_{3-y}^\infty{\frac{3}{2}x^2} dx
                = \int_{3-y}^1{\frac{3}{2}x^2} dx
                = \eval{\frac{x^3}{2}}{3-y}{1}
                = \frac{1 - (3-y)^3}{2}
            \]
            So:
            \[
                F(y) =
                \begin{cases}
                    0                     & \text{for } y \leq 2\\
                    \frac{1 - (3-y)^3}{2} & \text{for } 2 \leq y \leq 4\\
                    1                     & \text{for } 4 \leq y
                \end{cases}
            \]
            $f(y) = \frac{d}{dy}(\frac{1 - (3-y)^3}{2}) = \frac{3}{2} (3-y)^2$.
            Therefore:
            \fbox{
                \(
                    f(y) =
                    \begin{cases}
                        \frac{3}{2} (3-y)^2 & \text{for } 2 \leq y \leq 4\\
                        0                   & \text{elsewhere}
                    \end{cases}
                \)
            }
        
        \subsection{Find the density function of $Y = X^2$.}
            By method of distributon functions:\\
            $-1 \leq X \leq 1$, so $0 \leq X^2 \leq 1$, and $0 \leq Y \leq 1 $.\\
            $F(y) = P(Y \leq y) = P(X^2 \leq y) = P(-\sqrt{y} \leq X \leq \sqrt{y})$
            \[
                = \int_{-\sqrt{y}}^{\sqrt{y}}{\frac{3}{2}x^2} dx
                = \eval{\frac{x^3}{2}}{-\sqrt{y}}{\sqrt{y}}
                = y^{3/2}
            \]
            So:
            \[
                F(y) =
                \begin{cases}
                    0       & \text{for } y \leq 0\\
                    y^{3/2} & \text{for } 0 \leq y \leq 1\\
                    1       & \text{for } 1 \leq y
                \end{cases}
            \]
            $f(y) = \frac{d}{dy}(y^{3/2}) = \frac{3}{2} \sqrt{y}$.
            Therefore:
            \fbox{
                \(
                    f(y) =
                    \begin{cases}
                        \frac{3}{2} \sqrt{y} & \text{for } 0 \leq y \leq 1\\
                        0                    & \text{elsewhere}
                    \end{cases}
                \)
            }

    \newpage
    \section{Assume that $X$ has a beta distribution with parameters $\alpha$ and $\beta$.
        Find the density function of $Y = 1 - X$.}
        By method of transformations:\\
        We know $Y = 1 - X$, and $0 \leq X \leq 1$.\\
        It's easy to see that $0 \leq Y \leq 1$, and that $Y$ is decreasing\\
        $y = f(x) = 1 - x$, So $x = f^{-1}(y) = 1 - y$.
        \begin{align*}
            F_Y(y) &= F_X(1 - y) \\
            f_Y(y) &= f_X(1 - y) \abs{\frac{d(1-y)}{dy}}\\
                   &= f_X(1 - y) \abs{-1}
        \end{align*}
        Therefore: 
        \fbox{
            \(
                f(y) =
                \begin{cases}
                    \frac{(1 - y)^{\alpha - 1} y^{\beta - 1}}{B(\alpha, \beta)} & \text{for } 0 \leq y \leq 1
                    \text{, where $B(\alpha, \beta)$ is the Beta function.}\\
                    0                                                           & \text{elsewhere}
                \end{cases}
            \)
       }

    \section{$X$ is a uniformly-distributed random variable between $0$ and $1$.}
        \subsection{Find the probability density function of $Y = -\lambda \ln{X}$.}
            By method of transformations:\\
            $X = \exp(\frac{Y}{-\lambda})$. $Y$ is decreasing if $\lambda$ is positive, 
            increasing if $\lambda$ is negative, and undefined PDF if $\lambda = 0$. 
            \begin{align*}
                0 \leq &X \leq 1\\
                0 \leq &\exp(\frac{Y}{-\lambda}) \leq 1\\
                \ln 0 \leq &\frac{Y}{-\lambda} \leq \ln 1\\
                &\frac{Y}{-\lambda} \leq 0\\
                0 \leq &Y
            \end{align*}
            \(
                f_Y(y) = f_X(\exp(\frac{y}{-\lambda})) \abs{\frac{d(\exp(\frac{y}{-\lambda}))}{dy}}
                = f_X(\exp(\frac{y}{-\lambda})) \abs{\frac{\exp(\frac{y}{-\lambda})}{-\lambda}}
                = \frac{\exp(\frac{y}{-\lambda})}{\abs{\lambda}}
            \)
            \begin{center}
                Therefore: 
                \fbox{
                    \(
                        f(y) =
                        \begin{cases}
                            \frac{\exp(\frac{y}{-\lambda})}{\abs{\lambda}} & \text{for } y \geq 0,
                            \lambda \neq 0 \\
                            0                                                           & \text{elsewhere}
                        \end{cases}
                    \)
                }
            \end{center}
        
        \subsection{Find the expected value and standard deviation of $Y$.}
            $f(y)$ is the exponenial distribution.
            Expected value:
            \fbox{
                \(
                    \mu = \lambda
                \)
            }.
            Standard deviation:
            \fbox{
                \(
                    \sigma = \lambda
                \)
            }.


    \newpage
    \section{
        The lifetime of an electronic component in an HDTV is a random variable that can be
        modeled by the exponential distribution with a mean lifetime $\beta$. Two components, $X_1$
        and $X_2$, are randomly chosen and operated until failure. At that point, the lifetime of
        each component is observed. The mean lifetime of these two components is
        \[\mx = \frac{X_1 + X_2}{2}\]
    }
        \subsection{Find the probability density function of $\mx$ using the MGF technique (the method of moment-generating functions.)}
            By method of moment-generating functions: 
            \(
                m_{\mx}(t) = E(e^{t{\mx}}) = E(\exp(t \frac{x_1 + x_2}{2}))
                = E(\exp(t \frac{x_1}{2})\exp(t \frac{x_2}{2}))
                = E(\exp(t \frac{x_1}{2}))E(\exp(t \frac{x_2}{2})) \text{(because they are independent.)}
                = m_{x_1}(t/2)m_{x_2}(t/2)
                = (1 - \beta \frac{t}{2})^{-1} (1 - \beta \frac{t}{2})^{-1} = (1 - \beta \frac{t}{2})^{-2}
            \)\\
            $(1 - \beta \frac{t}{2})^{-2}$ is the MGF of the Gamma function with $\alpha '= 2, \beta ' = \beta / 2$.\\
            Therefore:
            \fbox{
                \(
                    f(\mx) = 
                    \begin{cases}
                    \frac{\mx^{2-1} e^{- \mx / (\beta / 2)}}{\Gamma(2) (\beta / 2)^2}
                    = \frac{4 \mx e^{- 2 \mx / \beta}}{\beta^2} &\text{for } x \geq 0\\
                    0 &\text{elsewhere}
                    \end{cases}
                \)
            }
        
        \subsection{
            If the mean lifetime of the electronic component is two years ($\beta = 2$), what is the
            probability that the mean lifetime of two tested components will be more than three
            years? You may use R and/or additional software to calculate your answer to this
            question, but please show the relevant equation(s) used to obtain the final answer.}
            We know $f(\mx) = \frac{4 \mx e^{- 2 \mx / \beta}}{\beta^2}$, and $\beta = 2$.
            So $f(\mx) = \mx e^{- \mx}$.\\
            \fbox{$P(\mx \geq 3) = 
                \int_3^\infty{\mx e^{ - \mx}} d \mx = 4 / e^3 \approx 19.915 \%$}

    \newpage
    \section{Let $X_1 , X_2 , \hdots , X_{10}$ repensent a sample of size $10$ taken from a normal distribution with
        $\mu = 0$ and $\sigma^2 = 1$. Define the following quantity:
        \[U = X^2_1 + X^2_2 + \hdots + X^2_{10}\]
        Find the distribution of $U$ and state the mean and standard deviation of $U$ as well.}
        By method of moment-generating functions: 
        \(
            m_{u}(t) = E(e^{t{u}}) = E(\exp(t (x^2_1 + x^2_2 + \hdots + x^2_{10})))\\
            = E(\exp(t x^2_1)) E(\exp(t x^2_2)) \hdots  E(\exp(t x^2_{10})) \text{(because they are independent.)}\\
            = m_{x^2_1}(t) m_{x^2_2}(t) \hdots m_{x^2_{10}}(t) = (m_{x^2}(t))^{10} \text{(because they are the same distribution.)}
        \)\\
        Let $\sigma' = \sqrt{- \frac{1}{2t - 1}}, \mu' = 0$.
        \begin{align*}
            m_{x^2}(t) 
            &= \int_{-\infty}^\infty{e^{t x^2} \frac{e^{-\frac{(x - 0)^2}{2 (1^2)}}}{1 \sqrt{2\pi}} } dx\\
            &= \frac{1}{\sqrt{2\pi}} \int_{-\infty}^\infty{e^{t x^2} e^{-\frac{x^2}{2}}} dx\\
            &= \frac{1}{\sqrt{2\pi}} \int_{-\infty}^\infty{e^{t x^2 -\frac{x^2}{2}}} dx\\
            &= \frac{1}{\sqrt{2\pi}} \int_{-\infty}^\infty{e^{x^2 (t -\frac{1}{2})}} dx\\
            &= \frac{\sigma'}{\sigma'} \frac{1}{\sqrt{2\pi}} 
            \int_{-\infty}^\infty{e^{(x - 0)^2 \frac{1}{2 \frac{1}{2 (t -\frac{1}{2})}}}} dx\\
            &= \sigma' \int_{-\infty}^\infty{\frac{e^{-\frac{(x - \mu')^2}{2 \sqrt{-\frac{1}{2t - 1}}^2}}}
            {\sigma' \sqrt{2\pi}}} dx \text{ (because imaginary numbers.)}\\
            &= \sigma' \int_{-\infty}^\infty{\frac{e^{-\frac{(x - \mu')^2}{2 \sigma'^2}}}
            {\sigma' \sqrt{2\pi}}} dx\\
            &= \sigma' = \sqrt{- \frac{1}{2t - 1}}
        \end{align*}
        $(m_{x^2}(t))^{10} = \sqrt{- \frac{1}{2t - 1}}^{10} = - \frac{1}{(2t - 1)^{5}} = \frac{1}{(1 - 2t)^{5}}
        = (1 - 2t)^{-5}$. Which is the MGF of the Gamma distribution with $\alpha = 5, \beta = 2$.\\
        \fbox{
            \(
                \mu = 10,\  \sigma = \sqrt{20},\  f(u) =
                \begin{cases}
                \frac{u^{4} e^{-u/2}}{768}
                &\text{for } 0 \leq u\\
                0 &\text{elsewhere}
                \end{cases}
            \)
        }

    \newpage
    \section{Let $X_1 , X_2 , \hdots , X_n$ represent a sample of observations taken from an exponentially
        distributed population with parameter $\beta$. Let $Y = X_1 + X_2 + \hdots + X_n$. Assuming the
        observations are independent random variables, identify the distribution of the random
        variable defined as
        \[Z = \frac{2Y}{\beta}\]}
        By method of moment-generating functions: 
        \(
            m_{y}(t) = E(e^{t{y}}) = E(\exp(t (x_1 + x_2 + \hdots + x_n))) \\
            = (m_x(t))^{n} \text{ (same reasoning as question 5)}
        \)
        \\
        Now. \(
            m_z(t) = E(e^{tz}) = E(e^{(t 2 y) / \beta}) = m_y(2t/\beta) = (m_x(2t/\beta))^{n} 
            = ((1 - \frac{\beta 2 t}{\beta})^{-1})^{n} = (1 - 2t)^{-n}
        \)
        \\
        Which is the MGF of the Chi-square distribution with $v = 2n$.\\
        \begin{center}
            \fbox{
                \(
                    \mu = 2n,\  \sigma = \sqrt{4n},\  f(z) =
                    \begin{cases}
                    \frac{z^{n - 1} e^{-z/2}}{2^n \Gamma(n)}
                    &\text{for } 0 \leq z^2\\
                    0 &\text{elsewhere}
                    \end{cases}
                \)
            }
        \end{center}


\end{document}

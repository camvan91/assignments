\documentclass[10pt, letterpaper, titlepage]{article}

%Size of section header
\usepackage{sectsty}
\sectionfont{\fontsize{12}{15}\selectfont}

%quattrocento font
\usepackage[sfdefault]{quattrocento}

\usepackage{amsmath}
\usepackage{xcolor}

%Header
\usepackage[margin=1in]{geometry}
\usepackage{fancyhdr}
\setlength{\headheight}{23.01503pt}
\pagestyle{fancy}
\lhead{}
\rhead{Yifeng Pan
     \\UCID: 30063828}

%Change lable to letter from number
\renewcommand{\thesubsection}{\alph{subsection}}

%Evaluate for calc
\newcommand*\eval[3]{\left.#1\right\rvert_{#2}^{#3}}

%Absolute Value
\newcommand\abs[1]{\left|#1\right|}

%Title page
\title{STAT 323 Assignment 4}
\author{Instructor: Claudia Marie Mahler
    \\Name: Yifeng Pan
    \\UCID: 30063828}
\date{Summer 2019}

\newcommand{\mx}{\overline{x}}
\newcommand{\my}{\overline{y}}
\newcommand{\mz}{\overline{z}}
\newcommand{\mX}{\overline{X}}

%For displaying R code
\usepackage{listings}

\usepackage{amssymb}
\newcommand{\Z}{\mathbb{Z}}
\newcommand{\R}{\mathbb{R}}

\newcommand{\E}{\text{E}}
\newcommand{\RE}{\text{RE}}
\newcommand{\B}{\text{B}}
\newcommand{\Var}{\text{Var}}
\newcommand{\Cov}{\text{Cov}}
\newcommand{\pv}{\text{p-value}}

\begin{document}
    \maketitle

    %P1
    \section{Let $X_1$ and $X_2$ constitute a random sample from a population which is normally distributed
        with $\sigma^2 = 1$. If the null hypothesis $\mu = \mu_0$ is to be rejected in favor of the alternative 
        hypothesis $\mu = \mu_a$, when $\mx = \mu_0 + 1$, what is the size of the critical region (rejection 
        region)? Assume $\mu_a > \mu_0$.}
        $\sigma = 1, \mx = \mu_0 + 1, P(R\ H_0 | H_0) = \alpha$\\
        $\frac{\mx - \mu_0}{\sigma}\sqrt{n} = \frac{\mu_0 + 1 - \mu_0}{\sqrt{1}}\sqrt{2} = \sqrt{2}$\\
        $1 - pnorm(\sqrt{2}) \approx 7.865\%$.

    %P2
    \section[]{Let $X_1$ represent a random sample of size $1$ from a population having the following
        probability density function:
        \[
            f(x) = 
            \begin{cases}
                \theta x ^{(\theta - 1)} & 0<x<1\\
                0 & \text{elsewhere}
            \end{cases}
        \]
        If the critical region $x_1 \geq 0.75$ is used to test the null hypothesis $\theta = 1$ against the
        alternative hypothesis $\theta = 2$, what is the power of the test at $\theta = 2$?}
        $P(R H_0 | H_a) = p$\\
        $P(x_1 > 0.75 | \theta = 2) = \int_{0.75}^1{2x}dx = 43.75 \%$

    %P3
    \section{A Calgary physician wants to determine whether a weight-reducing drug has a different effect on adults over $40$ 
        years of age than on adults that are no more than $40$ years of age. 
        For individuals who are no more than $40$ years of age, it is known that the mean weight loss on this drug 
        is $\mu=11.3$ pounds. Twelve people over the age of $40$ are given the drug; the mean weight loss is $8.9$ pounds with a sample 
        standard deviation of $4.1$ pounds. Does the data suggest that the drug has a different effect on adults over $40$ years 
        of age compared to adults that are no more than 40 years of age? Conduct a hypothesis test using $\alpha=0.05$.
        }
        $H_0: \mu_{\leq 40} = \mu_{> 40}$\\
        $H_a: \mu_{\leq 40} \neq \mu_{> 40}$\\
        $t = \frac{8.9-11.3}{4.1}\sqrt{12} \approx -2.028$\\
        $\pv = pt(t, 11) * 2 \approx 6.75 \% > 5\%$\\
        Therefore the $H_0$ is accepted.

    \newpage
    %P4
    \section{It has been suggested that the price of condominiums in Calgary have increased in the
        past six months. In order to test this claim, a sampling design was employed where the 
        selling price of $10$ randomly selected condos were chosen from both January and July of 
        this year. The selling prices are in terms of $\$1000$s and the data is in the {\color{red} Condo.R} 
        data file. Assuming that the selling price of condos are normally distributed, does the data 
        support the claim that the mean selling price of a condo in July is greater than the selling price of 
        a condo in January of this year? Use $\alpha = 0.05$.}
        $H_0: \mu_{Ju} \leq \mu_{Ja}$\\
        $H_a: \mu_{Ju} > \mu_{Ja}$\\
        R code:
        
        %The backslash is not defined in Quattrocento, so it throws two warnings.
        %Should be safe to ignore
        \lstinputlisting[language=R, firstline = 2, lastline = 18, numbers = left]{Condo.R}

        \noindent
        Output:
        \begin{verbatim}
            p-value for p4, first method:  0.4159381 
            p-value for p4, second method:  0.4159381 
        \end{verbatim}

        \noindent
        $\pv \approx 41.59\% > 5\%$\\
        Therefore the $H_0$ is accepted.

    \newpage
    %P5
    \section{A precision instrument is stated to have a measurement variation of no more than $0.49$
        units. A sample of four instrument readings on the same object yielded the measurements 
        $351.4, 351, 351.9,$ and $350.3$. Does this data suggest that the measurement variation of 
        the instrument is at most the stated $0.49$ units? Use $\alpha = 0.07$.}
        $H_0: \sigma^2 > 0.49$\\
        $H_a: \sigma^2 \leq 0.49$\\
        R code:

        %Same warning as above
        \lstinputlisting[language=R, firstline = 21, lastline = 24, numbers = left]{Condo.R}

        \noindent
        Output:
        \begin{verbatim}
            p-value for p5:  0.2874183
        \end{verbatim}
        
        \noindent
        $\pv \approx 28.74\% > 7\%$\\
        Therefore $H_0$ is accepted.

    %P6
    \section{A major court case on the health effects of drinking contaminated water took place in the town of Picture Butte, 
        Alberta. A town well in Picture Butte was contaminated with fecal bacteria due to run-off from a local cattle feedlot. 
        During the period that residents drank water from this well, there were $16$ birth defects among $414$ births. 
        In years when the contaminated well in question was not used and water was supplied from clean wells, 
        there were two birth defects among $228$ births. 
        The plaintiffs suing thefeedlot responsible for the water contamination claimed that these data show 
        that the rate of birth defects was higher when the contaminated well was in use. 
        Conduct a hypothesis test with $\alpha=0.01$ to determine if the rate of birth defects was significantly higher when the 
        contaminated well was in use.}
        $H_0: p_c \leq p$\\
        $H_a: p_c > p$\\
        $z = \frac{\hat p - p}{\sqrt{p(1-p)}}\sqrt{n}
            = \frac{16/414 - 2/228}{\sqrt{2/228 (1 - 2/228)}}\sqrt{414}
            \approx 6.519$\\
        $\pv = 1 - pnorm(z) \approx 3.539 \times 10^{-11} << 0.01$\\
        Therefore $H_0$ is rejected.

    \newpage
    %P7
    \section{Let $X_1 , X_2 , \hdots , X_n$ represent a random sample from a normal distribution where the value of 
        the mean $\mu$ is unknown and the variance is $\sigma^2 = 1$. Derive the uniformly most powerful test 
        criterion with $\alpha = 0.05$ used to test the hypotheses 
        \[ H_0 : \mu = 0 \]
        \[ H_a : \mu = 1 \]
        }
        \begin{align*}
            k > \frac{L(\theta_0)}{L(\theta_a)} 
            &= \prod_{i=1}^n \frac{\frac{1}{1 \sqrt{2\pi}}\exp(-(\frac{(x_i - 0)^2}{2}))}
                {\frac{1}{1 \sqrt{2\pi}}\exp(-(\frac{(x_i - 1)^2}{2}))}
            = \prod_{i=1}^n \frac{\exp(-(\frac{(x_i - 0)^2}{2}))}
                {\exp(-(\frac{(x_i - 1)^2}{2}))}
            = \prod_{i=1}^n \exp(-(\frac{(x_i - 0)^2}{2}) + (\frac{(x_i - 1)^2}{2}))\\
            &= \prod_{i=1}^n \exp(\frac{1}{2} ((x_i - 1)^2-x_i^2))
            = \prod_{i=1}^n \exp(\frac{1}{2} (1 - 2x_i))
        \end{align*}
        \begin{align*}
            \ln(k) > \ln(\frac{L(\theta_0)}{L(\theta_a)})
            &= \ln(\prod_{i=1}^n \exp(\frac{1}{2} (1 - 2x_i)))
            = \sum_{i=1}^n \frac{1}{2} (1 - 2x_i)
            = \sum_{i=1}^n \frac{1}{2} - x_i
            = \frac{n}{2} - \sum_{i=1}^n x_i
            = \frac{n}{2} - n \mx\\
            \ln(k) + n\mx &> n/2\\
            \mx &> \frac{n/2 - \ln(k)}{n}
                =\frac{1}{2} - \frac{\ln(k)}{n}\\
            \frac{\mx - \mu}{\sigma}\sqrt{n} &> \frac{(\frac{1}{2} - \frac{\ln(k)}{n}) - 0)}{\sqrt{1}}\sqrt{n}
                =\frac{\sqrt{n}}{2} - \frac{\ln(k)}{\sqrt{n}}
        \end{align*}
        (Note: The following approach is needlessly convoluted, but it avoids having to introduce a new varible.)
        \begin{align*}
            1 - pnorm(\frac{\sqrt{n}}{2} - \frac{\ln(k)}{\sqrt{n}}) &= 0.05\\
            pnorm(\frac{\sqrt{n}}{2} - \frac{\ln(k)}{\sqrt{n}}) &= 0.95 \\
            \frac{\sqrt{n}}{2} - \frac{\ln(k)}{\sqrt{n}} &= qnorm(0.95) \\
        \end{align*}
        Solve for $k$ and we get $k = \exp(\frac{n}{2} - qnorm(0.95)\sqrt{n})$.\\
        Therefore the test criterion for $\mu$ is:
        \begin{align*}
            \mx &> \frac{1}{2} - \frac{\ln(k)}{n}
            = \frac{1}{2} - \frac{\ln(\exp(\frac{n}{2} - qnorm(0.95)\sqrt{n}))}{n}\\
            &= \frac{1}{2} - \frac{\frac{n}{2} - qnorm(0.95)\sqrt{n}}{n}\\
            &= \frac{1}{2} -( \frac{1}{2} - qnorm(0.95)/\sqrt{n}))\\
            &= \frac{qnorm(0.95)}{\sqrt{n}} \approx \frac{1.645}{\sqrt{n}}
        \end{align*}
\end{document}